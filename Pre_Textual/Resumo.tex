% resumo em português
\setlength{\absparsep}{18pt} % ajusta o espaçamento dos parágrafos do resumo
\begin{resumo}
Os sistemas de recomendação tem sido muito estudados nos últimos anos devido a sua capacidade de filtrar a grande quantidade de informação da internet para melhorar o acesso dos usuários em aplicações web. Porém, a maior parte dos sistemas de recomendação utilizam apenas um critério no momento da recomendação, a acurácia da recomendação, mas essa  abordagem é limitada podendo gerar recomendações enviesadas. Sendo assim, nesse trabalho, será proposto  um sistema de recomendação baseado em conteúdo para recomendação de filmes que utiliza mais outros dois critérios além da acurácia: diversidade e novidade da recomendação, com isso, a recomendação foi modelada como um problema multiobjetivo. Portanto, neste sistema, a recomendação será feita a partir da execução de um algoritmo multiobjetivo para solucionar o problema de recomendação. Esse algoritmo fornecerá como resultado, um conjunto de soluções, mas, como não haverá garantia que essas soluções pertencem ao conjunto de itens reais, serão recomendados os filmes mais similares as soluções resultantes. O sistema será avaliado utilizando os \textit{datasets} de filmes Movielens e IMDB e seus resultados serão comparados com sistemas de recomendação do estado da arte.
 
\textbf{Palavras-chave}: sistemas de recomendação. otimização multiobjetivo. aprendizado de máquina. recomendação de filmes.
\end{resumo}