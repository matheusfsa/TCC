\chapter{Conclusão}

Neste trabalho foi apresentado o  MOEA-RS, que é um sistema de recomendação para filmes baseado em conteúdo auxiliado por um algoritmo de otimização multiobjetivo que tem como propósito encontrar filmes que otimizem  três medidas, acurácia, diversidade e novidade.

Para definir as técnicas que seriam utilizadas nos componentes do método, foram realizados uma série de experimentos com o propósito de comparar diversas técnicas de aprendizado de máquina e extração de características. Após a definição das técnicas, o método foi avaliado no \textit{dataset} MovieLens, com 1 milhão de avaliações de usuários sobre filmes. Os resultados do MOEA-RS foi comparado aos resultados de outras técnicas da literatura.

Para avaliar os resultados, foram utilizadas quatro medidas: precisão, \textit{recall}, diversidade e novidade. Como foi apresentado nos experimentos, o MOEA-RS obteve melhores resultados que os outros métodos com relação as  medidas de precisão, diversidade e novidade, e só não obteve melhor resultado, para a medida de \textit{recall}, que um dos métodos.

Portanto, pode-se concluir, que o MOEA-RS é um método com qualificado para recomendação de filmes, pois conseguiu superar métodos da literatura em medidas de qualidade tradicionais, como a precisão, e em medidas de qualidade que tem agregar muito valor em uma recomendação, como diversidade e novidade. Além disso, o MOEA-RS, em comparação com outros da literatura, é eficiente tanto na questão temporal quanto na espacial.

Porém, o método ainda pode ser melhorado, principalmente  a partir do uso de diferentes técnicas de extração de características e aprendizado de máquina. Outros algoritmos de otimização de objetivo podem ser utilizados e avaliados. Além disso, o MOEA-RS também pode ser avaliado em recomendação de outros itens.


%\lipsum[31-33]