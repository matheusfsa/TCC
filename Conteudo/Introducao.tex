\chapter{Introdução}
A internet já é realidade para uma grande parcela da população, em 2018, o número de usuário passou de 4 bilhões~\cite{internetusers}. Como consequência, a quantidade de informação que circula na internet também cresce com grande velocidade, a perspectiva é que, em 2020, o volume de dados na internet seja aproximadamente 40 trilhões de Gigabytes~\cite{volumedados}. Com isso, diversas ferramenta tem sido desenvolvidas para facilitar a busca de informações de um usuário na internet, com destaque para os sistemas de recomendações.

Sistemas de recomendações são técnicas que fornecem sugestões de itens para um usuário\cite{ricci2011introduction}. Essas técnicas são bastante utilizadas em aplicações que lidam com uma grande quantidade informações e que precisam filtrar uma parte dessas informações para melhorar a navegação do usuário na aplicação, algumas aplicações que utilizam essas ferramentas são o \textit{You Tube}, \textit{Netflix}, \textit{Amazon} e etc.

Normalmente, essas aplicações buscam recomendar uma lista de itens. Existem dois tipos de sistemas de recomendações, baseado em conteúdo e com filtro colaborativo, o primeiro busca recomendar itens semelhantes aos itens que usuário gostou anteriormente e o segundo busca recomendar itens com base nas avaliações de usuário com gostos parecidos.

Os sistemas de recomendações tradicionais levam em consideração apenas um critério no momento da recomendação, que é acurácia, ou seja, a chance do usuário gostar do item. Porém essa metodologia pode ser considerada limitada, já que ela não considera que o usuário pode utilizar mais que um critério ao fazer uma escolha\cite{adomavicius2011multi}. 

Muitos trabalhos já buscam fazer recomendações utilizando mais de um critérios, mas, grande parte busca agregar os critérios em um único critério~\cite{yager1988ordered}, \cite{kaymak1994selecting}. Esse tipo de abordagem apresenta limitações, pois a composição pode ocasionar em perda de informações.

Neste trabalho  foi definido um nova abordagem para sistemas de recomendações, onde esse tipo de problema é tratado como um problema multiobjetivo, que é um problema onde se deseja otimizar dois ou mais critérios. Para isso dois outros critérios, além da acurácia, são considerados: diversidade e novidade. A diversidade está relacionado ao quanto os itens da listas de recomendação diferem entre si, o propósito do uso desse critério, é fazer com que a recomendações não sejam monotemáticas\cite{hurley2011novelty}. A novidade é o quanto os itens recomendados são diferentes do que o usuário conhece\cite{hurley2011novelty}. 


Nesse problema, o propósito é encontrar um item que otimizem os três critérios definidos, como esses critérios são conflitantes, não existirá apenas um item que ótimo, mas um conjunto de itens, esse conjunto corresponderá a lista de recomendações que será fornecida como resultado para o usuário. Para encontrar esse conjunto de itens, será utilizado uma algoritmo evolucionário multiobjetivo(MOEA, em inglês), que se destacam pela capacidade de gerar um conjunto de soluções próximas ao ótimo  em apenas uma execução.

Vale ressaltar, que o espaço de itens não é continuo, já que o \textit{dataset} contem um número limitado de itens, então, por questão de eficiência, o MOEA será executado utilizando um espaço de busca continuo, onde qualquer combinação de valores de características(levando em consideração o mínimo e máximo das características), será uma solução válida. Portanto, não  há garantia que as soluções geradas pelo MOEA são itens do \textit{dataset}, sendo assim, os itens recomendados correspondem aos itens mais similares as soluções resultantes.

Esse trabalho tem como objetivo apresentar e aplicar essa  nova abordagem de sistemas de recomendação em um cenário especifico, que é a recomendação de filmes, esse tipo de cenário foi escolhido devido a sua complexidade, já que pela grande quantidade de itens(filmes) disponíveis, a recomendação baseada apenas na acurácia é limitada. O sistema proposto corresponde a um sistema de recomendação baseado em conteúdo, esse tipo de sistema foi escolhido, pois, como definido anteriormente, pois a solução do MOEA corresponde uma representação de um item. Esse método será avaliado utilizando \textit{dataset} tradicionais e seus resultados serão comparados com sistemas de recomendação do estado da arte.

Esse trabalho foi organizado da seguinte forma: na Seção 2 será descrita a fundamentação teórica utilizada no desenvolvimento do trabalho, na Seção 3 é apresentado o método proposto no trabalho e na Seção 4 é descrita as etapas para execução do trabalho.

